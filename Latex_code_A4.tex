%%%%%%%%%%%%%%%%%%%%%%%%%%%%%%%%%%%%%%%%%%%%%%%%%%%%%%%%%%%%%%%
%
% Welcome to Overleaf --- just edit your LaTeX on the left,
% and we'll compile it for you on the right. If you open the
% 'Share' menu, you can invite other users to edit at the same
% time. See www.overleaf.com/learn for more info. Enjoy!
%
%%%%%%%%%%%%%%%%%%%%%%%%%%%%%%%%%%%%%%%%%%%%%%%%%%%%%%%%%%%%%%%


% Inbuilt themes in beamer
\documentclass{beamer}

\providecommand{\pr}[1]{\ensuremath{\Pr\left(#1\right)}}
\providecommand{\qfunc}[1]{\ensuremath{Q\left(#1\right)}}
\providecommand{\sbrak}[1]{\ensuremath{{}\left[#1\right]}}
\providecommand{\lsbrak}[1]{\ensuremath{{}\left[#1\right.}}
\providecommand{\rsbrak}[1]{\ensuremath{{}\left.#1\right]}}
\providecommand{\brak}[1]{\ensuremath{\left(#1\right)}}
\providecommand{\lbrak}[1]{\ensuremath{\left(#1\right.}}
\providecommand{\rbrak}[1]{\ensuremath{\left.#1\right)}}
\providecommand{\cbrak}[1]{\ensuremath{\left\{#1\right\}}}
\providecommand{\lcbrak}[1]{\ensuremath{\left\{#1\right.}}
\providecommand{\rcbrak}[1]{\ensuremath{\left.#1\right\}}}
% Theme choice:
\usetheme{CambridgeUS}

% Title page details: 
\title{Assignment 3 \\ CBSE Class 11 Exe 16.3 ,16} 
\author{Burra Vishal Mathews \\ CS21BTECH11010}

\date{\today}
\logo{\large \LaTeX{}}


\begin{document}

% Title page frame
\begin{frame}
    \titlepage 
\end{frame}

% Remove logo from the next slides
\logo{}


% Outline frame
\begin{frame}{Outline}
\tableofcontents
    
\end{frame}
\section{Question}
\begin{frame}{Question}

If x is N(l000; 400) find (a) P\{x$<$1024\}, (b) P\{x$<$1024 x$>$961\}, and (c) P\{31 $<$ $\sqrt{x}$ $ \leq$ 32\}.

\end{frame}

\begin{frame}{Solution }
\section{Solution}

     x is in Normal distribution N($\mu$,$\sigma^2$) will be use to represent the Gaussian p.d.f
     \begin{block}
     
        \begin{align}
        F_x(x)=\int_{-\infty}^x \frac{1}{\sqrt{2\pi \sigma^2}}e^{-(y-\mu)^2/2\sigma^2}dy =G(\frac{x-\mu}{\sigma})
        \end{align}
     \end{block}
     
     $\mu$ = 1000\\
     $\sigma$ = 20\\

\end{frame}

\begin{frame}{Solution-part-a}

\textbf{(a)} 
\begin{align*}
    P\cbrak{ x<1024} &= G\brak{\frac{1024-1000}{20}}\\&=G(1.2)\\&=0.8849\\
\end{align*}

\end{frame}


\begin{frame}{Solution-part-b$\And$c}
\textbf{(b)}
\begin{align*}
    P\cbrak{x<1024 \vert x > 961} & =\brak{\frac{P\cbrak{961<x<1024}}{P\cbrak{x>961}}}\\
& =\brak{\frac{G(1.2)-G(-1.95)}{1-G(-1.95)}}\\
 & =0.8819\\ 
\end{align*}
\textbf{(c)}
    \begin{align*}
        P\cbrak{31<\sqrt{x}\leq 32}=P\cbrak{961<x\leq1024}=0.8593
    \end{align*}
\end{frame}


\end{document}